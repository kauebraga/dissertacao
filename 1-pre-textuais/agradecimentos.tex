
Ao Prof. Dr. Felipe Loureiro, por todos esses anos de ensinamentos e muita paciência. Mesmo que de forma inconstante, nossa caminhada começou lá em 2009, ainda quando eu estava no primeiro ano da minha graduação em Engenharia Civil. Desde então, sempre o admirei, e agora, com esse dois últimos anos de muitas conversas e orientações, admiro mais ainda. A sua sabedoria e clareza de ideias permitiu que esse trabalho tomasse a forma que tem hoje. Obrigado por tudo!

Aos amigos do Seridó's Garden, pelas frequentes risadas que guiaram grande parte do meu caminho nesse mestrado. Mesmo que distantes nos últimos meses, ainda tenho uma prazer enorme de estar ao lado de vocês mesmo que por alguns dias por mês. Obrigado pela cabuetagem!

Aos amigos e colegas do MITUS, que me adotaram e me guiaram durante boa parte do meu mestrado. Obrigado por todas as tardes de discussões sobre acessibilidade! 

Ao Dr. Rafael Pereira, por esses meses de uma convivência cheia de aprendizados. Essa dissertação teve uma grande reviravolta nos últimos meses, e grande parte dessa se deve aos seus ensinamentos e motivação para trabalhar com acessibilidade e dados de GTFS. Agradeço-o ainda por ter cedido os dados e os computadores potentes do IPEA para o desenvolvimento desse trabalho. Tem sido um prazer tê-lo como chefe (parceiro de trabalho, na verdade). Obrigado!

À minha namorada Marina, que me acompanhou de perto nessa por vezes dura caminhada. A distância imposta nos últimos meses tem sido um desafio frequentemente difícil de superar, mas a sua presença, mesmo que online, me motivou nesse processo. Tenho certeza que esse trabalho não seria o mesmo sem você na minha vida. Te amo! 

Aos meus pais, Carlos e Delmira, e à minha irmã Caliandra. Vocês são as pedras fundamentais da minha vida. Devo todo o meu caráter e minha conduta a vocês. Amo muito vocês. 



