Accessibility, defined as the potential of opportunities for interaction, is a key concept in urban planning. Traditionally, the measurement of accessibility relied on methods that used simplified information on transit networks. Recent developments of public transport structured data through Global Transit Feed Specification (GTFS) fostered the creation and standardization of many tools to estimate travel times between origin-destination pairs, feeding the calculation of accessibility indicators. However, researchers are beginning to better understand to what extent differences between scheduled GTFS data and real services can affect accessibility estimates. The aim of this dissertation is to analyze the variability of accessibility indicators by public transport to work and education opportunities using location data and smart card data in a large transportation system. This variability of accessibility is investigated through three main hypotheses: 1) there is a difference between measuring it with Scheduled GTFS versus Empiric GTFS;  2) there is a difference between measuring it with Empiric GTFS based on median observed travel times versus based on a dispersion measure of observed travel times on the network; 3) there is a difference between measuring it to different departure times within peak hour. Initially, the proposed method consolidates and integrates one month of big data database, preparing them to the GTFS reconstruction approach. Using geoprocessing techniques, it starts by transforming raw GPS points to a structured transit timetable. The timetable times are then aggregated by 15 minutes intervals and segments between stops, determining the median and 85 percentile travel time to produce two empiric GTFS files. These files, alongside the Scheduled GTFS, are then used to estimate travel time between origin-destination pairs, making possible to estimate accessibility indicators based on scheduled and real-time data at different departure times within peak hour. Smart card data is used to estimate time thresholds to calculate cumulative indicators. The results corroborate the three hypotheses of variability on the spatial distribution of accessibility indicator by work and education activities. This variability is higher for access to jobs opportunities given its strong concentration on the central area of Fortaleza. These results suggest that decision makers take this variability into consideration when they assess the accessibility impacts of future interventions.

% Separe as Keywords por ponto
\keywords{Public Transport. Big data. Accessibility. GTFS.}