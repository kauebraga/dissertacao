A acessibilidade, definida como o potencial de oportunidade de interação no espaço urbano, é um conceito chave para o planejamento das cidades. Tradicionalmente, sua mensuração dependeu de métodos que utilizavam informações simplificadas e/ou modeladas da rede de transporte público, trazendo consigo grande incerteza. Recentemente, a produção de dados estruturados da oferta de transporte público através de smart card, GPS e GTFS permitiu a criação e padronização de diversas ferramentas para a estimação de tempos de viagens entre pares origem-destino, alimentando assim o cálculo de indicadores de acessibilidade; reconhecendo-se, entretanto, que dados de GTFS podem diferir bastante da realidade operacional, com estudos anteriores destacando o uso do GPS na atualização empírica do GTFS. O Sistema Integrado de Transportes de Fortaleza (SIT-FOR), por sua vez, apresenta no seu big data de transporte público (dados de bilhetagem e GPS) uma grande potencialidade para utilização na estimação da acessibilidade às atividades realizadas diariamente por seus usuários. Esse trabalho de dissertação, portanto, se propôs a analisar a variabilidade de indicadores de acessibilidade por transporte público a oportunidades de trabalho e educação com o uso dados empíricos de localização da frota e dados de bilhetagem, no contexto da avaliação de intervenções no sistema de transportes de uma cidade de grande porte como Fortaleza. A variabilidade de indicadores de oportunidades acumuladas é analisada a partir de três hipóteses, assumindo um possível efeito sobre a sensibilidade da tomada de decisão: 1) há diferença entre calculá-los com GTFS programado versus GTFS corrigido pela tendência central dos tempos de viagem observados em campo; 2) há diferença entre calculá-los com GTFS corrigido com base em tempos de viagem medianos versus incorporando sua dispersão; 3) há diferença em calculá-los para diferentes horários de partida dentro do período de pico. Inicialmente, o método proposto consolida e integra as bases de big data, preparando-as para a aplicação da abordagem de reconstrução empírica do arquivo de GTFS. Com o uso de técnicas de geoprocessamento, esta abordagem começa por traduzir localizações dos veículos em tabelas de horários programados; esses horários são, então, transformados em tempos de viagem entre paradas e são agrupados por intervalo de 15 minutos em cada trecho, determinando-se sua mediana e seu 85o. percentil para a produção de dois arquivos de GTFS corrigido. Esses arquivos, por sua vez, são utilizados para o cálculo do tempo de viagem total entre pares origem-destino através de uma ferramenta de escolha de rota, possibilitando o cálculo dos indicadores de acessibilidade cumulativa para distintos arquivos GTFS e horários de partida. Dados de bilhetagem são utilizados para estimar o tempo limite para o cálculo do indicador de acessibilidade cumulativo. Para o caso de Fortaleza, observam-se evidências que corroboram as três hipóteses de variação nas distribuições espaciais dos indicadores de acessibilidade por tipo de atividade, sendo maior para oportunidades de trabalho que educação, dada a forte concentração da oferta de empregos na região central da cidade. Recomenda-se aos tomadores de decisão que levem em conta essa variabilidade quando das análises de sensibilidade em avaliações de intervenções em transportes que utilizem métricas de acessibilidade. 


% Separe as palavras-chave por ponto
\palavraschave{Transporte Público. Big Data. Acessibilidade. GTFS.}